%Jennifer Pan, August 2011

\documentclass[10pt,letter]{article}
	% basic article document class
	% use percent signs to make comments to yourself -- they will not show up.
\usepackage{gensymb}
\usepackage{amsmath}
\usepackage{amssymb}
	% packages that allow mathematical formatting

\usepackage{graphicx}
	% package that allows you to include graphics

\usepackage{setspace}
	% package that allows you to change spacing

%\onehalfspacing
	% text become 1.5 spaced

%\newcommand{\latex}{\LaTeX\xspace}
%\newcommand{\tex}{\TeX\xspace}
	% package that specifies normal margins
	
	
	
	\newcommand{\intt}{\int_{t_0}^{t}}
	\newcommand{\tn}{t_0}
	\newcommand{\Rin}{R_{\textrm{in}}}
	\newcommand{\Rout}{R_{\textrm{out}}}
	\newcommand{\Cin}{C_{\textrm{in}}}
	\newcommand{\So}{S_0}
	\newcommand{\Vto}{V(t_0)}
	\newcommand{\Vmax}{V_{\textrm{max}}}
	\newcommand{\trm}[1]{\textrm{#1}}
	

\usepackage{nopageno}
\usepackage{mathtools}
\DeclarePairedDelimiter\abs{\lvert}{\rvert}
\DeclarePairedDelimiter\prn{\left(}{\right)}
% use * or \abs[\Big]{} to fill space
\usepackage{enumerate}
\usepackage{natbib}
\usepackage{graphicx}
\usepackage{bm}
\usepackage{xfrac} 
\usepackage[ margin = 1in]{geometry}
\usepackage{textcomp}
%\usepackage{unicode-math}
\usepackage{titlesec}
%\setmainfont{Times New Roman}
%\setmathfont{Cambria Math}


\begin{document}
	% line of code telling latex that your document is beginning


\title{Stat Theory HW 2}

\author{Jordan Mandel}

\date{\today}
	% Note: when you omit this command, the current dateis automatically included
 
\maketitle 

\section*{1.}
\subsubsection*{Expectation:}
\[
E(X) 
= \sum_{x\in\mathbb{Z}}xf_X(x)
= \sum_{x=1}^\infty x\frac{-(1-p)^x}{x\log(p)}
=-\left(\frac{1}{\log(p)}\sum_{x=1}^\infty (1-p)^x\right)
=-\left(\frac{(1-p)}{\log(p)}\sum_{x=0}^\infty (1-p)^x\right)
=\frac{-(1-p)}{p\log(p)}
\]
\subsubsection*{Variance:}
$Var(X) = E(X^2)-(E(X))^2$.  We already calculated $E(x)$ so we just have to calculate $E(X^2)$, and plug it in to the formula.
\[
E(X^2) 
= \sum_{x\in\mathbb{Z}}x^2f_X(x)
= \sum_{x=1}^\infty x^2\frac{-(1-p)^x}{x\log(p)}
=-\left(\frac{1}{\log(p)}\sum_{x=1}^\infty x(1-p)^x\right)
=-\left(\frac{(1-p)}{\log(p)}\sum_{x=1}^\infty x(1-p)^{(x-1)}\right)
\]
To calculate the sum we note that for $r\in (0,1)$
\begin{align*}
\frac{d}{dr}\left(\frac{1}{1-r}\right)&= \frac{d}{dr}\left(\sum_{x=0}^\infty r^x\right) \\
\frac{-1}{(1-r)^2} &= \sum_{x=0}^\infty xr^{(x-1)}\textrm{,}
\end{align*}
and that $(1-p)$ is precisely such an $r$, and that the $x=0$ term is itself zero. So plugging this in  we have

\begin{equation*}
E(X^2) = \frac{(1-p)}{p^2log(p)}
\end{equation*}
And plugging in values into the main variance equation we get: 
\begin{equation*}
Var(X) = \left(\frac{(1-p)}{p^2log(p)}\right)^2-\frac{(1-p)}{p^2log(p)}=\boxed{\frac{(1-p)^2-(1-p)p^2\log{p}}{p^4\log(p)}}
\end{equation*}

\section*{2: CB 2.1}
\section*{3: CB 2.18}
\subsection*{(a)}
\subsection*{(b)}
\section*{4: CB 3.1}
\section*{5: CB 3.2}
\subsection*{(a)}
\subsection*{(b)}
\section*{6.}
\subsection*{(a)}
\subsection*{(b)}
\section*{7.}The we use the formula 
\begin{equation*}
f_Y(y)= f_X(g^{-1}(y))\abs*{\frac{d}{dy}(g^{-1}(y))}
\end{equation*}

\end{document}
